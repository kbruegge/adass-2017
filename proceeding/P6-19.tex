% This is the ADASS_template.tex LaTeX file, 26th August 2016.
% It is based on the ASP general author template file, but modified to reflect the specific
% requirements of the ADASS proceedings.
% Copyright 2014, Astronomical Society of the Pacific Conference Series
% Revision:  14 August 2014

% To compile, at the command line positioned at this folder, type:
% latex ADASS_template
% latex ADASS_template
% dvipdfm ADASS_template
% This will create a file called aspauthor.pdf.}

\documentclass[11pt,twoside]{article}

% Do not use packages other than asp2014.
\usepackage{asp2014}

\aspSuppressVolSlug
\resetcounters

% References must all use BibTeX entries in a .bibfile.
% References must be cited in the text using \citet{} or \citep{}.
% Do not use \cite{}.
% See ManuscriptInstructions.pdf for more details
\bibliographystyle{asp2014}

% 1 author: "Surname"
% 2 authors: "Surname1 and Surname2"
% 3 authors: "Surname1, Surname2, and Surname3"
% >3 authors: "Surname1 et al."
% Use mixed case type for the shortened title
% Ensure shortened title does not cause an overfull hbox LaTeX error
% See ASPmanual2010.pdf 2.1.4  and ManuscriptInstructions.pdf for more details
\markboth{Bruegge et al.}{Towards Robotic Operation with the First G-APD Cherenkov Telescope}

\begin{document}

\title{Towards Robotic Operation with the First G-APD Cherenkov Telescope}

% Note the position of the comma between the author name and the
% affiliation number.
% Author names should be separated by commas.
% The final author should be preceded by "and".
% Affiliations should not be repeated across multiple \affil commands. If several
% authors share an affiliation this should be in a single \affil which can then
% be referenced for several author names.
% See ManuscriptInstructions.pdf and ASPmanual2010.pdf 3.1.4 for more details
\author{Kai Br\"ugge$^1$  and Alexey Egorov$^2$}
\affil{$^1$TU Dortmund, Dortmund, Germany; \email{kai.bruegge@tu-dortmund.de}}
\affil{$^2$TU Dortmund, Dortmund, Dortmund; \email{alexey.egorov@tu-dortmund.de}}


% This section is for ADS Processing.  There must be one line per author.
\paperauthor{Kai~Brügge}{kai.bruegge@tu-dortmund.de}{}{TU Dortmund}{Experimental Physics 5}{Dortmund}{NRW}{44227}{Germany}
\paperauthor{Alexey~Egorov}{alexey.egorov@tu-dortmund.de}{}{TU Dortmund}{Artificial Intelligence Group}{Dortmund}{NRW}{44227}{Germany}

\begin{abstract}

  Once completed, the  Cherenkov Telescope Array (CTA)  will be able to map the gamma-ray sky in a wide energy range from several tens of GeV to some hundreds of TeV and will be more sensitive than previous experiments by an order of magnitude.
  It opens up the opportunity to observe transient phenomena like gamma-ray bursts (GRBs) and flaring active galactic nuclei (AGN).  In order to successfully trigger multi-wavelengths observations of transients, CTA has to be able to alert other observatories as quickly as possible.  Multi wavelength observations are essential for gaining insights into the processes occurring within these sources of such high energy radiation.

  CTA will consist of approximately 100 telescopes of different sizes and designs.
  Images are streamed from all the telescopes into a central computing facility on site.
  During observation CTA will produce a stream of up to 15 000 images per second. Noise suppression and feature extraction algorithms are applied to each image in the stream as well as previously trained machine learning models.
  Restricted computing power of a single machine and the limits of network's data transfer rates become a bottleneck for stream processing systems in a traditional single-machine setting.
  We explore several different distributed streaming technologies
  from the Apache Big-Data eco-system like Spark, Flink, Storm to handle the large amount of data coming from the telescopes.
  To share a single code base while executing on different streaming engines we employ  abstraction layers such as the streams-framework and the Apache Beam project.
  These use  a high level language to build up processing pipelines that can transformed into native the pipelines of the different platforms.
  Here we present results of our investigation and show a first prototype capable of analyzing CTA data in real-time.

\end{abstract}

\section{The Cherenkov Telecope Array}
Bla Bla Eien zitation hier bitteschoen: \cite{pmml}. Lecker Bild hier: \ref{fig:auc}

\articlefigure{P6-19_f1.eps}{fig:auc}{Sie moegen gerne EIER.}
%
% \begin{figure}[htbp]
%   \begin{center}
%     \includegraphics[width=\textwidth]{P6-19_f1.eps}
%     \caption{\label{fig:auc} \citep{pmml}}
%   \end{center}
% \end{figure}



\bibliography{P6-19}

\end{document}
